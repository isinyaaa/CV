%!TeX program = pdflatex
\documentclass{resume}

\usepackage{tikz, lmodern, amsmath, amssymb}

\definecolor{papercolor}{HTML}{282828}
\definecolor{textcolor}{HTML}{eeeeee}

\RequirePackage[pagecolor={papercolor}]{pagecolor}
\color{textcolor}


\usepackage{fontawesome5}

\newcommand{\bars}[1]{
    \tikz[overlay, remember picture] \foreach\i in {1,...,3}
        \draw ({(\i - 2)*4pt},0) rectangle +(2pt,{(\i + 1)*2 pt});
    \tikz[overlay, remember picture] \foreach\i in {1,...,#1}
        \fill ({(\i - 2)*4pt - 2.5pt},0) rectangle +(2pt,{(\i + 1)*2 pt});
}

\begin{document}

\fontfamily{ppl}\selectfont

\noindent
\begin{tabularx}{\linewidth}{@{}m{0.78\textwidth} m{0.2\textwidth}@{}}
{
    \Large{Isabella Basso do Amaral} \newline
    \small{
        \clink{
            \href{mailto:isabellabdoamaral@usp.br}{\faIcon{envelope}\ isabellabdoamaral@usp.br} \textbf{·}
            \href{https://wa.me/5511995100304}{\faIcon{whatsapp}}
            \href{https://telegram.me/isinyaaa}{\faIcon{telegram}}
            {\fontdimen2\font=0.75ex +55 11 995 100 304}
            \textbf{·}
            \href{https://github.com/isinyaaa}{\faIcon{github} @isinyaaa}
        } \newline
          São Paulo, Brazil\newline
          {\footnotesize avid Linux user \faIcon{linux} \textbf{·} open-source \faIcon{heart}}
        %{\footnotesize *Rio de Janeiro durante a quarentena}
    }
} &
{
    \centering
    \frame{\includegraphics[width=3cm]{images/photo.png}}
}
\end{tabularx}
\begin{center}
\begin{tabularx}{\linewidth}{@{}*{2}{X}@{}}
% left side %
{
    \csection{Work experience (mainline)}{\small
        \begin{itemize}
            \item \frcontent{Backend Developer}
            {Data Machina}
            {Python, Google Cloud, Firebase, Docker, Git}
            {
                I develop a WhatsApp chatbot;

                Team lead;

                I mentor a fellow intern and assist in various maintenance
                tasks.
            }
            {February 2021 onwards}
            \item \frcontent{Programmer}
            {Freelancer}
            {Python \& Java}
            {
                I've delivered many programming exercises.
            }
            {July 2020 up to January 2021}
        \end{itemize}
    }
    \csection{Education}{\small
        \begin{itemize}
            % item 1 %
            \item \frcontent{Molecular Sciences BSc\newline emphasis on maths \& computer science}
            {Universidade de São Paulo}
            {Ongoing}
            {
                as a natural scientist, I strive in interdisciplinary contexts;
                \href{http://cecm.usp.br/}{that's exactly what my degree is about}.

                scientific guidance student:

                - currently learning about algorithms and optimization;

                - previous topic was \href{https://github.com/isinyaaa/wimpa}{graph theory}.
            }
            {from 2020 onwards}
            %\item \frcontent{Associate degreee in Data Science}
            %{Universidade Estácio de Sá}
            %{Ongoing}
            %{
            %    I started studying data science topics to sharpen my
            %    understanding and improve my skills.
            %}
            %{from 2021 onwards}
            \item \frcontent{Physics BSc}
            {Universidade de São Paulo}
            {Interrupted}
            {
                the degree I started at;

                as a passionate student I used to wonder about the HOWs;

                then I found a degree more aligned to my goals.
            }
            {2020}
            \item \frcontent{Industrial Automation Technical School}
            {Instituto Federal Fluminense}
            {Concluded}
            {
                organized \href{http://lattes.cnpq.br/9507659630401803}{several projects and scientific presentations};

                I was my class' top student
            }
            {2017 -- 2019}
        \end{itemize}
    }
}
% end left side %
&
% right side %
{
    \csection{Projects (side quests)}{\small
        \begin{itemize}
            \item \frsimple{Linux sys-admin}
            {Tech}
            {Bash}
            %{
            %    Busco me qualificar como sys-admin por utilizar Linux no dia-a-dia e
            %    me interessar pelo funcionamento do sistema, assim como em utilizá-lo
            %    plenamente
            %}
            %{}
            %\item \frsimple{Technical illustration design}
            %{Art}
            %{\LaTeX}
            %{
            %    Produzi {\href{https://github.com/isinyaaa/werk-yago-pub}{material didático}} para diversas disciplinas
            %    de curso técnico;
            %
            %    Atualmente faço ilustrações técnicas para um
            %    projeto de {\href{https://petfisica.home.blog/olimpet/}{ensino de física}}
            %}
            %{}
            %\item \frsimple{Translation}
            %{English}
            %{english--portuguese}
            %{
            %    Traduzi diversas obras de
            %    filosofia para publicação (artigos e livro)
            %}
            %{}
            \item \frsimple
            {Algorithmic math problem resolutions \clink{\href{https://projecteuler.net/}{[projecteuler.net/]}}}
            {Maths}
            {Julia, Haskell \& Python}
            %{
            %    Resolvi diversos problemas do acervo para afiar minhas
            %    habilidades analíticas
            %}
            %{}
            \item \frsimple{Bare Windows' UIs \clink{\href{https://github.com/isinyaaa}{[github.com/isinyaaa]}}}
            {Tinkering}
            {C\# \& XAML}
            %{
            %    Fruto de um interesse de longa data em interfaces e interações para
            %    além do CLI
            %}
            %{}
        \end{itemize}
    }
    \csection{Skills}{\small
      	\begin{itemize}
      		\item \textbf{Programming}
      		{\footnotesize
      			\begin{itemize}
      			\item[{\bars{3}}] ($\geqslant$2y): \LaTeX\, (2.5y) \& Python (4y)
      			\item[{\bars{2}}] ($\geqslant$1y): Bash, C\# \& Java
      			\item[{\bars{1}}] ($<$1y): C/C++, Haskell, JavaScript, Julia \& Lua
      		\end{itemize}}
      		\item \textbf{Languages}
      		{\footnotesize
      			\begin{itemize}
      			\item[{\bars{3}}] Portuguese (native), English (C1)
      			\item[{\bars{1}}] German, Spanish \& French
      		\end{itemize}}
      	\end{itemize}
    }
}
\end{tabularx}
\end{center}
\end{document}