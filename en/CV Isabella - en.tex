\documentclass{resume}

\usepackage{tikz, lmodern, amsmath, amssymb}
\usepackage{fontawesome5}

\definecolor{papercolor}{HTML}{282828}
\definecolor{textcolor}{HTML}{eeeeee}

%\usepackage[pagecolor={papercolor}]{pagecolor}
%\color{textcolor}

\newcommand{\bars}[1]{
    \tikz[overlay, remember picture] \foreach\i in {1,...,3}
        \draw({(\i - 2)*4pt},0) rectangle + (2pt,{(\i + 1)*2 pt});
    \tikz[overlay, remember picture] \foreach\i in {1,...,#1}
        \fill({(\i - 2)*4pt - 3.5pt},0) rectangle + (2pt,{(\i + 1)*2 pt});
}

\begin{document}
\noindent
\begin{tabularx}{\linewidth}{@{}m{1\textwidth}}
    % {@{}m{0.78\textwidth} m{0.2\textwidth}@{}}
{
    \Large{Isabella Basso do Amaral} \newline
    \small{
        \clink{
            \href{mailto:isabellabdoamaral@usp.br}{\faIcon{envelope}\ isabellabdoamaral@usp.br}
            \href{https://telegram.me/isinyaaa}{\faIcon{telegram}}
          \href{https://gitlab.com/isinyaaa}{\faIcon{gitlab}}
            \href{https://github.com/isinyaaa}{\faIcon{github} @isinyaaa}
        } \newline
          São Paulo, Brazil\newline
        {\footnotesize \faIcon{linux} Linux enthusiast/independent contributor \textbf{·} \faIcon{heart} FLOSS}
        %{\footnotesize *Rio de Janeiro durante a quarentena}
    }
}
% &
% {
%    \centering
%    \frame{\includegraphics[width=3cm]{images/photo.png}}
% }
\end{tabularx}

\medskip
% \begin{flushleft}
{
    \small
    % As a young woman who's grown up in this century, I've lived through the
    % most innovative technological advancements in human history, but not
    % without a glooming feeling creeping through it.
    
    I hold a firm belief that we should build sustainable things to help
    maintain and improve what good things we've achieved so far as a species,
    and as a technophile, I feel the urge to help reshape the way technology is
    built and distributed.
    
    It's with that spirit that I started using and contributing to Free
    Software, which at this point has become part of a life mission.
    
    % As someone who's passionate about technology, living in my age I'm also
    % very concerned with sustainability, and that's what led me to Free Software.
}
% \end{flushleft}

\begin{center}
\begin{tabularx}{\linewidth}[t]{@{}*{2}{X}@{}}
    \begin{csection}{Education}
        \item \frcontent{Molecular Sciences BSc\newline emphasis on maths \& computer science}
        {July 2020 onwards}
        {University of São Paulo}
        {Ongoing}
        {
            As a scientist by nature, I thrive in interdisciplinary contexts;
            \href{http://cecm.usp.br/}{that's exactly what my degree is about}.

            In my degree:

            - I'm looking into research topics dealing with FPGAs
            (specifically high-level synthesis) optimizing use cases regarding
            OpenMP and using Free Software
            -- research project starts in July 2022.

            As a scientific guidance student:

            - I studied algorithms \& \href{https://github.com/isinyaaa/uni-latex/tree/main/combinatorics}{graph theory}.
        }
        %\item \frcontent{Physics BSc}
        %{Universidade de São Paulo}
        %{Interrupted}
        %{
        %    The degree I started at;
        %    As a passionate student I used to wonder about the HOWs;
        %    Then I found a degree more in line with my goals.
        %}
        %{2020}
        \item \frcontent{Industrial Automation Technical High School}
        {2017 -- 2019}
        {Federal Fluminense Institute}
        {Concluded}
        {
            As a top student:

            - I helped my friends in my free time and led class projects;

            - I organized several scientific projects.
            
            My academic CV (which has a list of project that I participated in):
            \clink{
                \href{http://lattes.cnpq.br/9507659630401803}
                {[lattes.cnpq.br/]}
            }
        }
    \end{csection}
&
    \vspace{-16.5pt}%
    \begin{csection}{Work experience}
        \item \frcontent{Software Engineer}
        {November 2021 onwards}
        {Red Hat}
        {Django, SQL, Python, Containers, Bash, C, Git, Remote team collaboration, CI}
        {
            I currently work with static analysis CI solutions for Fedora/RHEL RPM packages.
        }
        \item \frcontent{Internship}
        {February 2021 up to October 2021}
        {Data Machina}
        {Python, Google Cloud, Firebase, Docker, Git, Agile Methods, Augmented Reality (AR), Research}
        {
            Data Machina is a Brazilian startup company with a focus on data science solutions.

            As an intern:

            - I mentored another intern;

            - I explored the development of an AR solution;

            - and I also helped significantly in the development of a WhatsApp chatbot.

            - I was recognized by my peers as a leading talent.

            - I assisted and led various key maintenance tasks (deploying, setting up CI, etc.).
        }
    \end{csection}
\end{tabularx}
\end{center}

\begin{center}
\begin{tabularx}{\linewidth}[t]{@{}*{2}{X}@{}}
    \begin{csection}{Side projects}
        \item \frcontent{FLOSS group member}
        {June 2021 onwards}
        {LKCAMP \clink{\href{https://lkcamp.dev/}{[lkcamp.dev/]}}}
        {C, Git, Profiling Tools, Kernel, Linux}
        {
            LKCAMP is a group from Unicamp (the state university of Campinas) that has as its
            purpose to help students contribute to free software.
            
            As a hackathon participant:
            
            - I helped my group and engaged profusely with the community.
            
            Then, as a group member:
            
            - I gave a lecture about
            \href{
                https://www.youtube.com/watch?v=d5DutGFJhh0
            }{Kernel Device Drivers};

            - I also helped organize another hackathon.
        }
        \item \frcontent{Linux Kernel mentee}
        {July 2021 onwards}
        {Mentored by AMD GPU engineer Rodrigo Siqueira}
        {Bash, C, Git, Profiling Tools, Kernel}
        {
            Siqueira helped me in my journey as an independent Linux Kernel
            contributor.
            
            As a mentee:
            
            - I learned many things about the Linux GPU/DRM subsystem;
            
            - As part of my learning I fixed many compilation warnings in the
            AMD driver;
            
            - I also helped with internal code restructuring to make FPU code
            more maintainable;
            
            - I started a blog to share my experiences and knowledge.
            
            My blog:
            \clink{
                \href{https://isinyaaa.github.io}
                {[github.io/]}
            }
            
            List of upstream commits by me:
            \clink{
                \href{https://git.kernel.org/pub/scm/linux/kernel/git/torvalds/linux.git/log/?qt=author&q=Isabella+Basso}
                {[git.kernel.org/]}
            }
        }
        %\item \frsimple{Data Girls Neuron\clink{\href{https://www.linkedin.com/company/data-girls-neuron/}{[Linked In]}}}
        %{Data Science, events and community support}
        %{Member}
        % \item \frcontent{INOVA USP lab}
        % {Bash, Linux}
        % {Sys-admin}
        % \item \frsimple{Cryptography and Cybersecurity study club}
        % {Cryptography, Cybersecurity}
        % {Group member}
        %{
        %    Busco me qualificar como sys-admin por utilizar Linux no dia-a-dia e
        %    me interessar pelo funcionamento do sistema, assim como em utilizá-lo
        %    plenamente
        %}
        %{}
        %\item \frsimple{Technical illustration design}
        %{Art}
        %{\LaTeX}
        %{
        %    Produzi {\href{https://github.com/isinyaaa/werk-yago-pub}{material didático}} para diversas disciplinas
        %    de curso técnico;
        %
        %    Atualmente faço ilustrações técnicas para um
        %    projeto de {\href{https://petfisica.home.blog/olimpet/}{ensino de física}}
        %}
        %{}
        %\item \frsimple{Translation}
        %{English}
        %{english--portuguese}
        %{
        %    Traduzi diversas obras de
        %    filosofia para publicação (artigos e livro)
        %}
        %{}
        %\item \frsimple
        %{Algorithmic math problem resolutions \clink{\href{https://projecteuler.net/}{[projecteuler.net/]}}}
        %{Maths}
        %{Julia, Haskell \& Python}
        %{
        %    Resolvi diversos problemas do acervo para afiar minhas
        %    habilidades analíticas
        %}
        %{}
    \end{csection}
    &
    \vspace{-11pt}%
    \begin{csection}{Skills}
        \item \textbf{Programming}
        {\footnotesize
            \begin{itemize}
            \item[{\bars{3}}] ($\geqslant$2y): \LaTeX\, (3y) \& Python (4y)
            \item[{\bars{2}}] ($\geqslant$1y): Bash, C, C\# \& Java
            \item[{\bars{1}}] ($<$1y): C++, Haskell \& Rust
        \end{itemize}}
        \item \textbf{Languages}
        {\footnotesize
            \begin{itemize}
            \item[{\bars{3}}] Portuguese (native), English (C1 + 6mo interchange + translation experience)
            \item[{\bars{2}}] Spanish \& German
            \item[{\bars{1}}] French
        \end{itemize}}
    \end{csection}
\end{tabularx}
\end{center}

\end{document}