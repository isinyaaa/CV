\documentclass{resume}

\usepackage{tikz, amsmath, amssymb}

\newcommand{\bars}[1]{
    \tikz[overlay, remember picture] \foreach\i in {1,...,3}
        \draw({(\i - 2)*4pt},0) rectangle + (2pt,{(\i + 1)*2 pt});
    \tikz[overlay, remember picture] \foreach\i in {1,...,#1}
        \fill({(\i - 2)*4pt - 3.5pt},0) rectangle + (2pt,{(\i + 1)*2 pt});
}

\begin{document}

\noindent
\begin{tabularx}{\linewidth}{@{}m{1\textwidth}}
    % {@{}m{0.78\textwidth} m{0.2\textwidth}@{}}
{
    \Large{Isabella Basso do Amaral} \newline
    \small{
        \clink{
            \href{mailto:isabellabdoamaral@usp.br}{\faIcon{envelope}\ isabellabdoamaral@usp.br}
            \href{https://telegram.me/isinyaaa}{\faIcon{telegram}}
            \href{https://gitlab.com/isinyaaa}{\faIcon{gitlab}}
            \href{https://github.com/isinyaaa}{\faIcon{github} @isinyaaa}
        } \newline
            São Paulo, Brazil\newline
            she/her\newline
        {
            \footnotesize
            \faIcon{linux} Linux enthusiast/independent contributor \textbf{·} \faIcon{heart} OSS
        }
    }
}
% &
% {
%    \centering
%    \frame{\includegraphics[width=3cm]{photo.png}}
% }
\end{tabularx}

\medskip
% \begin{flushleft}
{
    \small
    
    I hold a firm belief that we should build sustainable things to help
    maintain and improve what good things we've achieved so far as a species,
    and as a technophile, I feel the urge to help reshape the way technology is
    built and distributed.

    I work and live by these principles, as a developer who's passionate about
    her work and mindful of the world we live in.
    
    % It's with that spirit that I started using and contributing to open source
    % software, which at this point has become part of a life mission.
    
    % As someone who's passionate about technology, living in my age I'm also
    % very concerned with sustainability, and that's what led me to Free Software.
}
% \end{flushleft}

\begin{center}
\begin{tabularx}{\linewidth}[t]{@{}*{2}{X}@{}}
    \begin{csection}{Education}
        \item \frcontent{Molecular Sciences BSc\newline emphasis on maths \& computer science}
        {University of São Paulo}
        {August 2020 onwards}
        {Ongoing}
        {
            As a scientist by nature, I thrive in interdisciplinary contexts;
            \href{http://cecm.usp.br/}{that's exactly what my degree is about}.

            In my degree:
            \begin{itemize}
                \item I'm trying to understand what makes some graphics APIs
                    great, and how we can match the convenience of closed
                    source ones with open source alternatives, improving their
                    adoption in scientific aplications like machine learning.
            \end{itemize}
            
            As a scientific guidance student:
            \begin{itemize}
                \item I studied algorithms \&
                    \href{https://github.com/isinyaaa/uni-latex/tree/main/combinatorics}{graph
                    theory};
                \item I studied and presented topics of high complexity in our
                    local study group.
            \end{itemize}
        }
        % \item \frcontent{Physics BSc}
        % {Universidade de São Paulo}
        % {2020}
        % {Interrupted}
        % {
        %   blablabla
        % }
        \item \frcontent{Industrial Automation Technical High School}
        {Federal Fluminense Institute}
        {2017 up to 2019}
        {Concluded}
        {
            As a top student:
            \begin{itemize}
                \item I helped my friends in my free time and led class
                    projects;

                \item I organized several scientific projects.
            \end{itemize}
            
            My academic CV (which has a list of project that I participated in):
            \clink{
                \href{http://lattes.cnpq.br/9507659630401803}
                {[lattes.cnpq.br/]}
            }
        }
    \end{csection}
&
    \vspace{-16.5pt}%
    \begin{csection}{Work experience}
        \item \frcontent{Software Engineer}
        {Red Hat}
        {November 2021 onwards}
        {Django, SQL, Python, Containers, Bash, C, Git, Remote team collaboration, CI}
        {
            I currently work with static analysis CI solutions for Fedora/RHEL RPM packages.
            
            As a software engineer:
            \begin{itemize}
                \item I'm updating a legacy Python codebase;

                \item I'm helping my team in upstreaming and open-sourcing
                    efforts (expected completion by the end of this year);
            
                \item I'm engaged in supporting our project's internal
                    (company-wide) userbase ($\sim 5000$ developers).
            \end{itemize}
        }
        \item \frcontent{Internship}
        {Data Machina}
        {February 2021 up to October 2021}
        {Python, Google Cloud, Firebase, Docker, Git, Agile Methods, Augmented Reality (AR), Research}
        {
            Data Machina is a Brazilian startup company with a focus on data science solutions.

            As an intern:
            \begin{itemize}
                \item I mentored another intern;

                \item I explored the development of an AR solution;

                \item I also helped significantly in the development of a
                    WhatsApp chatbot;

                \item I was recognized by my peers as a leading talent;

                \item I assisted and led various key maintenance tasks
                    (deploying, setting up CI, etc.).
            \end{itemize}
        }
    \end{csection}
\end{tabularx}
\end{center}

\begin{center}
\begin{tabularx}{\linewidth}[t]{@{}*{2}{X}@{}}
    \begin{csection}{Side projects}
        \item \frcontent{Google Summer of Code participant}
        {\clink{\href{https://summerofcode.withgoogle.com/}{[summerofcode.withgoogle.com]}}}
        {May 2022 up to September 2022}
        {C, Git, Kernel, Mailing lists}
        {
            I'm currently working on a project to improve unit test coverage in
            AMD's Linux kernel graphics driver.
            
            As a GSoC student:
            \begin{itemize}
                \item I've had to deal with production hardware code,
                    documenting, testing, and improving it;

                \item I'll present my work internally to AMD and also in an
                    international conference.
            \end{itemize}
            
            \mbox{My GSoC project:
            \clink{
                \href{https://summerofcode.withgoogle.com/programs/2022/projects/6AoBcunH}
                {[summerofcode.withgoogle.com/]}
            }}
            
            My talk:
            \clink{
                \href{https://indico.freedesktop.org/event/2/abstracts/78/}
                {[indico.freedesktop.org/]}
            }
        }
        \item \frcontent{Linux Kernel mentee}
        {Mentored by AMD GPU engineer Rodrigo Siqueira}
        {July 2021 onwards}
        {Bash, C, Git, Profiling Tools, Kernel}
        {
            Siqueira helped me in my journey as an independent Linux Kernel
            contributor.
            
            As a mentee:
            \begin{itemize}
                \item I learned many things about Linux' GPU subsystem (DRM);
            
                \item As part of my learning I fixed many compilation warnings
                    in AMD's driver;
            
                \item I also helped with internal code restructuring to make
                    FPU code more maintainable;
            
                \item I started a blog to share my experiences and knowledge.
            \end{itemize}
            
            My blog:
            \clink{
                \href{https://isinyaaa.github.io}
                {[github.io/]}
            }
            
            List of my upstream commits:
            \clink{
                \href{https://git.kernel.org/pub/scm/linux/kernel/git/torvalds/linux.git/log/?qt=author&q=Isabella+Basso}
                {[git.kernel.org/]}
            }
        }
        \item \frcontent{LKCAMP group member}
        {\clink{\href{https://lkcamp.dev/}{[lkcamp.dev]}}}
        {June 2021 onwards}
        {C, Git, Profiling Tools, Kernel, Linux}
        {
            LKCAMP is an OSS-focused group from Unicamp (the state university
            of Campinas) that has as its purpose to help students contribute to
            open source software.
            
            As a hackathon participant:
            \begin{itemize}
                \item I helped my group and engaged profusely with the
                    community.
            \end{itemize}

            Then, as a group member:
            \begin{itemize}
                \item I gave a lecture about
                    \href{
                        https://www.youtube.com/watch?v=d5DutGFJhh0
                    }{Kernel Device Drivers};

                \item I also helped organize another hackathon.
            \end{itemize}
        }
        %\item \frsimple{Data Girls Neuron\clink{\href{https://www.linkedin.com/company/data-girls-neuron/}{[Linked In]}}}
        %{Data Science, events and community support}
        %{Member}
        % \item \frcontent{INOVA USP lab}
        % {Bash, Linux}
        % {Sys-admin}
        % \item \frsimple{Cryptography and Cybersecurity study club}
        % {Cryptography, Cybersecurity}
        % {Group member}
        %{
        %}
        %{}
        %\item \frsimple{Technical illustration design}
        %{Art}
        %{\LaTeX}
        %{
        %}
        %{}
        %\item \frsimple{Translation}
        %{English}
        %{english--portuguese}
        %{
        %}
        %{}
        %\item \frsimple
        %{Algorithmic math problem resolutions \clink{\href{https://projecteuler.net/}{[projecteuler.net/]}}}
        %{Maths}
        %{Julia, Haskell \& Python}
        %{
        %}
        %{}
    \end{csection}
    &
    \vspace{-11pt}%
    \begin{csection}{Skills}
        \item \textbf{Programming}
        {\footnotesize
            \begin{itemize}
            \item[{\bars{3}}] ($\geqslant$2y): \LaTeX\, (3y) \& Python (4y)
            \item[{\bars{2}}] ($\geqslant$1y): Bash, C, C++, C\# \& Java
            \item[{\bars{1}}] ($<$1y): Haskell \& Rust
        \end{itemize}}
        \item \textbf{Languages}
        {\footnotesize
            \begin{itemize}
            \item[{\bars{3}}] Portuguese (native), English (C1 + 6mo interchange + translation experience)
            \item[{\bars{2}}] Spanish \& German
            \item[{\bars{1}}] French \& Finnish
        \end{itemize}}
    \end{csection}
\end{tabularx}
\end{center}

\end{document}
