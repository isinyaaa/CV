%!TeX program = pdflatex
\documentclass{resume}

\usepackage{tikz, amsmath, amssymb}
\usepackage{fontawesome5}

\newcommand{\bars}[1]{
    \tikz[overlay, remember picture] \foreach\i in {1,...,3}
        \draw ({(\i - 2)*4pt},0) rectangle +(2pt,{(\i + 1)*2 pt});
    \tikz[overlay, remember picture] \foreach\i in {1,...,#1}
        \fill ({(\i - 2)*4pt - 2.5pt},0) rectangle +(2pt,{(\i + 1)*2 pt});
}

\begin{document}

\fontfamily{ppl}\selectfont

\noindent
\begin{tabularx}{\linewidth}{@{}m{1\textwidth}@{}}%{@{}m{0.78\textwidth} m{0.2\textwidth}@{}}
{
    \Large{Isabella Basso do Amaral} \newline
    \small{
        \clink{
            \href{mailto:isabellabdoamaral@usp.br}{\faIcon{envelope}\ isabellabdoamaral@usp.br} \textbf{·}
            \href{https://wa.me/5511995100304}{\faIcon{whatsapp}}
            \href{https://telegram.me/isinyaaa}{\faIcon{telegram}}
            {\fontdimen2\font=0.75ex +55 11 995 100 304}
            \textbf{·}
            \href{https://github.com/isinyaaa}{\faIcon{github}}
            \href{https://gitlab.com/isinyaaa}{\faIcon{gitlab} @isinyaaa}
        } \newline
        São Paulo, Brasil\newline
        {\footnotesize *Rio de Janeiro durante a quarentena}\newline
        {\footnotesize Linux \faIcon{linux} enthusiast \textbf{·} \faIcon{heart} FOSS}
    }
} %&
%{
%    \centering
%    \frame{\includegraphics[width=3cm]{images/photo.png}}
%}
\end{tabularx}
\begin{center}
\begin{tabularx}{\linewidth}{@{}*{2}{X}@{}}
% left side %
{
    \csection{Experiência (mainline)}{\small
        \begin{itemize}
            \item \frcontent{Associate Software Engineer}
            {Red Hat}
            {C, Python, Git, Colaboração com equipe remota, CI}
            {
                Atualmente trabalho com soluções de integração contínua de análise estática para pacotes RPM do Fedora e do RHEL.
            }
            {November 2021 onwards}
            \item \frcontent{Backend Developer}
            {Data Machina}
            {Python, Google Cloud, Firebase, Docker, Git, Metodologia Ágil}
            {
                - Mentorava um estagiário;

                - Liderava o desenvolvimento de uma solução de realidade aumentada.

                - Antes desenvolvia um chatbot para WhatsApp;

                Referência para a equipe;

                Auxiliei e liderei diversas atividades chave de manutenção do
                produto (deploying, setup de CI, etc.).
            }
            {Fevereiro de 2021 até Outubro de 2021}
            \item \frcontent{Programadora}
            {Freelancer}
            {Python \& Java}
            {
                - Tutoria de computação;

                - Desenvolvi soluções para problemas científicos.
            }
            {Julho de 2020 em diante}
        \end{itemize}
    }
    \csection{Projetos (side quests)}{\small
        \begin{itemize}
            \item \frsimple{Grupo LKCAMP \clink{\href{https://lkcamp.dev/}{[lkcamp.dev/]}}}
            {C, Linux}
            {Membra do grupo, contribuidora independente do Kernel Linux}
            \item \frsimple{Mentoria de kernel}
            {C, Linux}
            {Mentorada por um engenheiro de GPU da AMD, contribuidora independente do Kernel Linux}
            %\item \frsimple{Data Girls Neuron \clink{\href{https://www.linkedin.com/company/data-girls-neuron/}{[Linked In]}}}
            %{Data Science, divulgação e apoio}
            %{Membra}
            %\item \frsimple{Grupo de estudos de criptografia}
            %{Criptografia}
            %{Membra}
            \item \frsimple{Lab do INOVA USP}
            {Bash, Linux}
            {Sys-admin}
            %{
            %    Busco me qualificar como sys-admin por utilizar Linux no dia-a-dia e
            %    me interessar pelo funcionamento do sistema, assim como em utilizá-lo
            %    plenamente
            %}
            %{}
            %\item \frsimple{Design de ilustrações técnicas}
            %{Art}
            %{\LaTeX}
            %{
            %    Produzi {\href{https://github.com/isinyaaa/werk-yago-pub}{material didático}} para diversas disciplinas
            %    de curso técnico;
            %
            %    Atualmente faço ilustrações técnicas para um
            %    projeto de {\href{https://petfisica.home.blog/olimpet/}{ensino de física}}
            %}
            %{}
            %\item \frsimple{Tradução}
            %{English}
            %{inglês--português}
            %{
            %    Traduzi diversas obras de
            %    filosofia para publicação (artigos e livro)
            %}
            %{}
            %\item \frsimple
            %{Problemas de computação matemática \clink{\href{https://projecteuler.net/}{[projecteuler.net/]}}}
            %{Haskell \& Python}
            %{}
            %{
            %    Resolvi diversos problemas do acervo para afiar minhas
            %    habilidades analíticas
            %}
            %{}
            %\item \frsimple{Construção de UI \clink{\href{https://github.com/isinyaaa}{[github.com/isinyaaa]}}}
            %{Tinkering}
            %{C\# \& XAML}
            %{
            %    Fruto de um interesse de longa data em interfaces e interações para
            %    além do CLI
            %}
            %{}
        \end{itemize}
    }
}
% end left side %
&
% right side %
{
    \csection{Educação}{\small
        \begin{itemize}
            % item 1 %
            \item \frcontent{B.S. Ciências Moleculares\newline Enfâse em matemática \& computação}
            {Universidade de São Paulo}
            {Cursando}
            {
                Cientista por natureza, busco a interdisciplinaridade nessa
                graduação desafiadora;

                Estudante de iniciação científica:

                - Atualmente interessada pelo \href{https://isinyaaa.github.io}{Kernel Linux e sua stack gráfica};
                %- Atualmente interessada em FPGAs \& otimização;

                - Anteriormente: \href{https://github.com/isinyaaa/uni-latex/tree/main/combinatorics}{teoria de grafos} \& algoritmos.
            }
            {2020 -- atual}
            %\item \frcontent{B.S. Física}
            %{Universidade de São Paulo}
            %{}
            %{Interrompido}
            %{2020}
            \item \frcontent{Curso técnico integrado em Automação Industrial}
            {Instituto Federal Fluminense}
            {Concluído}
            {
                Organizei \href{http://lattes.cnpq.br/9507659630401803}{diversos projetos e apresentações científicas};
                Destaque na turma
            }
            {2017 -- 2019}
        \end{itemize}
    }
    \csection{Habilidades}{\small
      	\begin{itemize}
      		\item \textbf{Programação}
      		{\footnotesize
      			\begin{itemize}
      			\item[{\bars{3}}] ($\geqslant$ 2y)  \LaTeX (2.5y) \& Python (4y)
      			\item[{\bars{2}}] ($\geqslant$ 1y)  Bash, C\# \& Java
      			\item[{\bars{1}}] ($<$ 1y)          C/C++, Haskell
      		\end{itemize}}
      		\item \textbf{Linguagens}
      		{\footnotesize
      			\begin{itemize}
      			\item[{\bars{3}}] Fluente: Português (nativa), Inglês (certificado C1)
                \item[{\bars{2}}] Intermediário: Espanhol \& Francês
      			\item[{\bars{1}}] Básico: Alemão
      		\end{itemize}}
      	\end{itemize}
    }
}
\end{tabularx}
\end{center}
\end{document}